\documentclass[a4paper, 11pt]{article}
\usepackage{graphicx}
\usepackage{natbib}
\usepackage{amsmath}
\usepackage[left=3cm,top=3cm,right=3cm]{geometry}

\renewcommand{\topfraction}{0.85}
\renewcommand{\textfraction}{0.1}
\parindent=0cm

%\title{Document Title}
%\author{Brendon J. Brewer}

\begin{document}
%\maketitle

The complex model implies a joint prior
\begin{eqnarray}
p_c(\theta, x) = p_c(\theta)p_c(x|\theta) = p_c(x)p_c(\theta|x).
\end{eqnarray}

The simple model implies its own joint prior
\begin{eqnarray}
p_s(\theta, x) = p_s(\theta)p_s(x|\theta) = p_s(x)p_s(\theta|x).
\end{eqnarray}

We would like to get the posterior under the complex model, $p_c(\theta|x)$,
but that might be too hard.
We wouldn't really mind if we could compute the results with the
simple model's prior. This would give a posterior
\begin{eqnarray}
p(\theta | x) \propto p_s(\theta)p_c(x|\theta).
\end{eqnarray}
but we can't compute that either because $p_c(x|\theta)$ is intractable.

One distribution for $\theta$ that we can compute is the following:
\begin{eqnarray}
q_\eta(\theta | x) \propto p_s(\theta)\mathcal{L}_\eta(\theta; x).
\end{eqnarray}
The normalised version of this distribution is
\begin{eqnarray}
q_\eta(\theta | x) &=& \frac{p_s(\theta)\mathcal{L}_\eta(\theta; x)}
{\mathcal{Z}(\eta; x)}
\end{eqnarray}
where $\mathcal{Z}(\eta; x) = \int p_s(\theta)\mathcal{L}_\eta(\theta; x) \, d\theta$.

Here $\mathcal{L}$ is some scalar that quantifies the match between $\theta$ and
$x$ and it also depends on adjustable parameter(s) $\eta$. A concrete example
would be saying that $\mathcal{L}_\eta(\theta; x) = p_s(x|\theta)^{1/\eta}$, i.e.
using a tempered version of the simple model's likelihood function.
The expected score (with a logarithmic scoring rule) when using this approximating
distribution is
\begin{eqnarray}
U &=& \int p_c(\theta | x) \log q_\eta(\theta | x) \, d\theta\\
&=& \int p_c(\theta | x) \log \left[\frac{p_s(\theta)\mathcal{L}_\eta(\theta; x)}
{\mathcal{Z}(\eta; x)}\right] \, d\theta
\end{eqnarray}
and we want to choose the value of $\eta$ that maximises this.

%We could do a MaxEnt type approximation, which IMO is not correct here because
%the constraint is only an approximation and not legitimate information. But it
%might be tractable.


\end{document}

